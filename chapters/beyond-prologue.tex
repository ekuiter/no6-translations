% !TeX spellcheck = en_US

Let me tell you a story. A story that I know. Story? No―it is reality,
humans will probably say. They will say it is reality engraved in human
history.

But for me, the deeds of humans are all but stories. At times a comedy,
at times a tragedy; sometimes predictable, sometimes wearisome―nothing
but fabrications.

Yes, humans are always but foolish actors.

They act out a farce, dancing at the mercy of their greed, love, and
emotions. They are foolish, ignorant, and avaricious.... They destroy
with their own hands what they have created. They aspire to rule over
others and become the one and only king of the world.

Why is that, I wonder?

Why are humans the only ones unable to live by the laws of nature,
leaving everything as is? They are such strange creatures.

In the story I am about to tell you now, the main character is also a
human―no. The main character is actually a city. A city-state. People
called it No. 6. Have you ever heard the name before? It is the most
beautiful, yet most fearsome, existence created by human hands. Worthy
of a star role in a farce, don't you think?

But... strange as it is, for some reason, I feel a sort of love towards
that city, No. 6. The story surrounding No. 6, as well those who have
lived in the story itself, are endearing to me. Does that make me the
possessor of a "soul"?

I know of two young boys.

Night and day; light and dark; earth and wind; one who embraces all, and
one who attempts to throw it all away. They are so different, yet they
are very much alike. Both were deeply involved with No. 6. They lived
their lives along with No. 6.

What? When was that, you say?

I wonder. It feels like only yesterday, but at the same time, it feels
like a thousand years ago. I do not feel time the way humans do.

I feel no difference between a single moment or an eternity.

But I have not forgotten about them.

Sometimes I feel that the chronicle of their lives is perhaps the only
one worth telling.

Come hither, now.

Let me tell you a story.

The story of two boys and of No. 6.