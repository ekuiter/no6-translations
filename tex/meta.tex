% !TeX spellcheck = en_US

\title{No. 6}
\author{Atsuko Asano}
\newcommand{\translator}{Nostalgia on 9\textsuperscript{th} Avenue}
\newcommand{\publisher}{Kodansha}
\newcommand{\coverauthor}{Toru Kageyama}
\newcommand{\booktitle}{}

\newcommand{\series}{
	\addbook{Part One}{Volume I -- III}
	\addbook{Part Two}{Volume IV -- VI}
	\addbook{Part Three}{
		Volume VII -- IX\\
		Beyond\\
		Side Stories\\
		Afterwords
	}
}

\newcommand{\structurenote}{
	First, the story was divided into three parts, each comprising several volumes of the novel, to have appropriate book-sized parts.
	In addition, the side stories were moved to part three, as they do not contribute to the main story.
	For similar reasons, the author's afterwords for each volume were collectively moved to the end of part three.
}

\newcommand{\longcopyright}{
	\vspace{0.3cm}\\
	\noindent \emph{\thetitle} is a nine-volume Japanese novel series written by \theauthor{} and published by \publisher{} between October 2003 and June 2011.
	It was subsequently adapted as a manga drawn by Hinoki Kino and as an anime television series produced by Bones.\vspace{2ex}\\
	The present edition is based on the English translations created by \translator{}, which can be found at \mysmallhref{http://9th-ave.blogspot.com/p/no-6.html}{9th-ave.blogspot.com/p/no-6.html}.
	All nine volumes of the novel are included; as well as the special \emph{Beyond} volume, the side stories \emph{Days in the West Block} (included with volume 4 of the manga) and \emph{Flowers for beautiful days} (volume 6 of the manga), and the author's afterwords found in several volumes.\vspace{2ex}\\
	As these translations are neither authorized nor licensed, please refrain from buying or selling this book.
	To support the author, consider buying the original novels or the manga.
	You can also contact Kodansha Children's Books at \mysmallhref{http://children.kodansha.co.jp/contact}{children.kodansha.co.jp/contact} to get \emph{\thetitle} published in English.\vspace{2ex}\\
	In preparing the present edition, a few decisions were made to adapt the translations to a printed book:
	\structurenote
	Regarding the text, not all of the many notes and references added by the translator are present; instead, only selected footnotes deemed relevant for understanding the story were included.
	At last, some changes regarding formatting and presentation were made to improve the reading experience.
}

\newcommand{\shortcopyright}{
	\vspace{2cm}\\
	\noindent \emph{\thetitle} is a nine-volume Japanese novel series written by \theauthor{} and published by \publisher{} between October 2003 and June 2011.
	The present edition is based on the English translations created by \translator{}, which can be found at \mysmallhref{http://9th-ave.blogspot.com/p/no-6.html}{9th-ave.blogspot.com/p/no-6.html}.\vspace{2ex}\\
	As these translations are neither authorized nor licensed, please refrain from buying or selling this book.
	To support the author, consider buying the original novels or the manga.
}

\newcommand{\colophon}{%
	This book was formatted by fans of the story.
	The text was set in 10½ point \href{https://ctan.org/tex-archive/fonts/urw/garamond/}{URW~Garamond~No.~8}.
	Headings, epigraphs and poems were set in \href{https://www.huertatipografica.com}{Alegreya~Sans}.
	Handwritten memos were set in \href{http://www.joebob.nl/}{Joe~Hand~2} (Karan), \href{http://www.dafont.com/david-kerkhoff.d2542}{Sunday~\&~Monday} (Nezumi), \href{http://www.ingofonts.com/}{Biro~Script} (Shion), \href{http://www.dafont.com/gabriel-de-ioannes-becker.d1961}{Gabo~4} (Inukashi), and \href{http://loremipsum.ro/}{Lipsum} (Sasori).
	The cover was created by \coverauthor{}.\vspace{2ex}\\
	Typesetting was done using \emph{\LaTeX}; the source code can be found at \myhref{https://github.com/ekuiter/no6-translations}{github.com/ekuiter/no6-translations}.
	Do feel free to report any typesetting mistakes.
	This book was built on \today.
}