% !TeX spellcheck = en_US

\clearpage
\subsection{Volume I}
\paragraph{Bunko}

Afterwords always make me terribly sheepish. It's embarrassing. Every
time I write one, somewhere in my heart, I shrink back from shame. I
hear my own voice telling me, \emph{how can you do such an embarrassing thing
with no hesitation?}

It probably comes from the fact I have used all my past afterwords as
excuses. And unconsciously, too, which makes it even worse. I've always
struggled to fill the gaping inadequacies of my work, somehow, with the
afterwords. I have a feeling that's what I've been trying to do.

After I realized what I was doing, I vowed not to write any more
afterwords. I thought that whatever a writer said or wrote outside of
his work was meaningless.

At the time of this writing, No.~6 has become a \emph{bunko}\footnote{\emph{bunko}: paperback edition, \emph{tankobon}: hardcover edition}.
Having been poor for a long time, as a reader, I can say I have a close
relationship with \emph{bunko}. This small and affordable book was a godsend to
my wallet and its meagre funds.

Thank you, \emph{bunko}.

So that being said, I can frankly say that I'm happy that this story has
become a \emph{bunko}, so that other people with meagre funds but a love for
books can have access to it. Whether it's worthy enough to read, well,
let's leave that judgment for another day. I have no choice but to leave
it in your hands, reader. I have no intention of saying things like,
"I've poured my life's effort into this"---those kind of words don't even
qualify as an excuse. I still want to believe that I haven't been
corrupted to that level.

The story isn't caught up with reality. It's very true. The things that
are portrayed in this story---tragedy, cruelty, the tyranny of those with
power, human greed, murderous intent\el take any one of these, and
you'll see that what you find in the world we live in far surpasses
anything told in my story.

How can humans be so cruel? So inhumane? It leaves me speechless in
shock. But despite being struck speechless, I ask myself, would I still
be able to find a hope for life through the story of \emph{No.~6}? The chances
of that seem uncertain, and slimmer than the contents of my wallet. But
I have no other way to do it but to write, and I feel like I would lose
to the cruelty and arrogance of reality---and I can't just put my tail
between my legs and admit defeat, so I write. I want to face off against
reality, approach it in challenge, with \emph{No.~6} as my strength. I want to
tear off that hide of what they call Reality or Human Beings, drag out
what lies beneath, and build upon it not despair, but a story of hope.

That is also my ambition.

Ah, am I making excuses again? Or am I just trying to cheer myself on?
Or am I brandishing valorous words to trick myself and others into
believing them? Hmm. That's really terrible, actually. But still\el 

\emph{You're annoying.}

I felt like I just heard Nezumi's whisper.

\emph{What an annoying woman. If you have time to be indulging yourself
in complaints, put up a fight first.}

I hear a voice telling me to fight, more stoutly, more fiercely than
anything---whether it be myself, or others, or the times. I grimace, and
give myself a shake.

He's right. For now, before writing an afterword, I'll write my story---a
story with no complaints, excuses or trickery.

So there you have it, an afterword that's not much of an afterword. I'm
really sorry. If I could, I would like to make this my last
afterword(-ish) thing.

So this is the end. I would like to extend my heartfelt thanks to those
in Kodansha's Children's Books Office: the late Mr. Yamakage Yoshikatsu,
Mr. Yamamuro Hideyuki, and Ms. Jinbo Junko from the Bunko Publishing
Department.

Thank you, thank you so much.

\myspace

\emph{2006, late summer\\
Atsuko Asano}

\clearpage
\subsection{Volume II}
\paragraph{Tankobon}

As you are reading this particular page of the story right now, what
sort of scene is unfolding around you?

What is happening with the wars, with starvation, with the world? Is the
killing still continuing? Is hatred still overflowing? Is despair still
brimming?

Do you believe in the word "hope"? I've always wanted to believe in
it---that the world could be mended, that people would be able to throw
their weapons aside. Someday.

Writing stories for young people is none other than to tell a tale of
hope---because there should be nothing born from despair.

That was how I've felt up until now, and obedient to that belief, I've
been weaving stories that tell of hope, but cavalierly.

"You don't know anything. You don't know what it's like to starve, to
shiver in the cold, to groan from a wound that's festered because it's
been left untreated too long; you don't know the suffering that follows
when that wound becomes infested with maggots, and you start rotting
alive; you don't know how it feels to watch someone die in front of you,
while there's nothing you can do to help them. You don't know a single
thing. You're just rattling off pretty words."

"You're just looking for an escape route. You're looking for a way to
avoid getting hurt."

"Words aren't things that you can toss around casually. You can't let
yourself be forced to say something, and just put up with it. But you
don't know that. So that's why I'm not going to trust you."

The numerous harsh words that Nezumi hurled at Shion were also blades
bared against me, and needles that stabbed my body.

Yes: I feel like I've lived thus far without knowing anything, nor
trying to know. I suffer no ailments; I never need to worry about food
for tomorrow; I live life without having to feel a smidgeon of fear from
being blasted by landmines or rocket bombs. I love my somewhat boring,
but peaceful life. And that's fine in itself. But when I peeled back a
bit of that peaceful life, I couldn't go without seeing that it was
actually very closely connected with foreign lands that seemed so
distant; with the war and starvation that people were suffering in those
lands.

Individuals are always connected to their nation, and the nation is
always connected to the rest of the world. It is impossible to cut them
apart. And I have finally realized that.

That was why I wanted to write this story, no matter what it took. Along
with a certain boy called Shion, I wanted to reach out and touch the
world. I wanted to write of a young and clumsy soul opening up his
physical body, and understanding the world through the pain and joy he
felt through it.

But to be honest, there were several times while writing when I thought
I would never be able to be like Shion. I couldn't face off with the
world as honestly as he. I couldn't yearn for another as earnestly. I
couldn't weave words as truthfully. And I was afraid of getting hurt. I
was always coming up with convenient excuses for myself. I couldn't beg
like he could.

At this point of having written up this story, for some reason I feel
something closer to defeat rather than fulfilment.

I'm sorry, here I go again, complaining. Those most unsteady in their
stance are the ones that talk the most, and make the most complaints.

Anyway, the story is still developing. I sincerely hope that you will be
able to enjoy it as Shion and Nezumi live, move, and weave their story
into existence.

I have no idea what will happen to these two, either. I'm not being mum
on purpose: I honestly can't predict what will happen.

But this is for certain: I do know that I don't want to leave Shion as
an idealist who is all talk; and I don't want to make Nezumi into a
terrorist of pure hatred. I would not want that to happen, no matter
what. So what do I need to do in order for it not to happen? What is
needed for them to survive, for them to avoid "becoming enemies", as
Nezumi once said? I know that I must think about this with a steady gaze
not on fantasy, but reality. And that must mean to focus the spotlight
on the ugliness of the nation-state, the frailty of human beings, my own
low-handedness, and never to avert that gaze.

And of course, in the end, I want to tell a tale of hope---not cavalierly,
with an agreeable smile on my face, using limp and lifeless words that
are merely pleasing to the ear. I want to speak with words I've invested
my own self into---I could mumble them, for what it's worth---but I want to
speak of hope, the kind I've grasped with my own hands. I want to become
that kind of writer.

I don't have the confidence I'll succeed. I already know very well how
powerless and incapable I am. But to me, it seems like there's still no
other way than to keep fighting alongside these young men.

I dedicate my heartfelt gratitude and hold in utmost admiration, Mr.
Yamakage Yoshikatsu of Kodansha's Editing Department, but at the same
time I want to complain to him, "It's so draining, this work." But I
know that he would probably---no, definitely---reply with, "You're being
indulgent. You're a professional. At least make sure you don't let
Nezumi and Shion laugh at you. Come on, straighten up."

Well, we've come to the end. My gratitude to the following people (no
complaints this time): Mr. Kageyama Toru, for creating the world of \emph{No.
6} more realistically, more fantastically, than anything my imagination
would have been able to create; and Mr. Kitamura Takashi, for giving No.
6 its unique glow and shadow through photos. Thank you.

\myspace

\emph{February 2004\\
Atsuko Asano}

\paragraph{Bunko}

To all of you who have read \emph{No.~6 \#2}: first of all, I send you my
thanks from the bottom of my heart.

This time, I decided to lend the narrative point of view to Shion, and
write from his place in the interior of the citadel city of No.~6,
looking out into the outside world of the West Block.

What sort of image did that place reflect in your eyes and hearts,
readers? By continuing to write this story, I am continually faced by my
own hypocrisy, which can be emotionally stressing sometimes\el no, all
the time. How can someone like me, who has never starved or froze, write
about people who live in the West Block?

If anything, it's arrogant and irresponsible; and for that reason I've
never liked to talk about this story, and if I force myself to open my
mouth, all that comes out is complaints and excuses. I'm sorry.

But still, to me, young men (and young women) of this age are
fascinating, and are figures that I have a profound attraction for. I so
badly want to know how they will live in this world, that instead of
learning from my mistakes, I arrogantly and irresponsibly continue to
write a story like \emph{No.~6}. As I hold both joy and fear in my heart that
this book will be seen and read by more people in its form as a \emph{bunko}, I
think I would like to live alongside these young men and women for just
a little longer.

Thank you very, very much for reading.

\myspace

\emph{February 2007\\
Atsuko Asano}

\clearpage
\subsection{Volume III}
\paragraph{Tankobon}

So how did you find \emph{No.~6} \#3? I know there's really no need to give
backstage-talk about the making of this novel, but\el will you listen
nonetheless?

To tell you the truth, before I began working on Volume 3, I was making
big promises to my editor Mr. Yamakage Yoshikatsu, telling him, "They're
going into the Correctional Facility now. It's going to be full of
action, I tell you, action." At this point, I wasn't lying or trying to
pique his interest. I was serious. After all, one of the motives I had
for writing \emph{No.~6} was my ambition to express thrilling action scenes
through words. But once I entered the world of Volume 3, and lived
alongside Shion and Nezumi, I realized it wasn't going to be as easy as
bursting into the Correctional Facility, causing a ruckus and then being
finished.

As I aligned my heart with theirs, wavered in uncertainty with them, and
mulled it over, sighing in despair or in awe, wondering why we fight,
why we love, why we hate, why we kill---my pages were up. It ended with no
big changes unfolding in the plot; no solving of puzzles; not even a
change in the season---it ended just when things seemed to be about to
begin. I know this, and others have said so too, that I am a person of
many excuses. But this time, I'm fully prepared to take complaints from
readers who will tell me, "What the heck is this?" and I will confess
that this time, I have no excuse to make.

But once inside the Correctional Facility, they will have to fight. The
possibilities are incredibly high that they will spill the blood of
others, or that their own blood will be spilt. If they had to end up
killing someone, or if one of them were to get killed, Shion and Nezumi
would have no choice but to undergo a change. A drastic change would
occur, not in the external sense, but to their young souls. I struggled
as I thought through how I would accept this reality, and how I would
write it, searching for an answer while I kept writing Volume 3.

I cannot forget reading the words of a certain adolescent, whom the
newspaper dismissed as a terrorist. He is said to have mumbled the
following to the hostages him and his group had captured: "What can I do
in order to be friends with you?"

I don't like war or terrorism. I despise it. And that is why I want to
know what sets him and his words apart from the rest of us. Whether I
have that power or not---it's not very clear, and honestly, I can't see
myself as having that sort of power. But I want to put up a fight. Part
of that fight is \emph{No.~6}, and this story. Ah, this is becoming an excuse
after all. Perhaps by the time the cherry blossoms have completely
fallen, I would be able to deliver you the rest of my struggle in the
form of Volume 4, as I place the focus on the two boys who had no choice
but to infiltrate the Correctional Facility. That will also be a fight
for me, where I put me and my excuse-prone self on the line. I extend my
heartfelt thanks to Mr. Yamakage for supporting my fight, and putting up
with my excuses so patiently; also I thank artist Mr. Kageyama Toru and
photographer Mr. Kitamura Takashi for expressing the world of \emph{No.~6} in
their own unique and creatively abundant ways, three times so far.

\myspace

\emph{October 2004\\
	Atsuko Asano}

\paragraph{Bunko}

Hello, everyone. Asano here. Thank you very much for accompanying me in
the world of \emph{No.~6}.

I would ask, how did you find it?---but a question like that is the
epitome of unsophistication. Let me seal it away.

It has been nearly three years since Volume 3 was first published. I'm
sure you would agree that these three years have been worthy to call
tumultuous. People's hearts, values, the state of society, and the
goings-on of our world have switched directions, mutated, and changed at
dizzying speeds.

Love, justice, the future---things we all believed in without question are
on the verge of disappearing without a trace. Maybe that's the kind of
world we live in now.

I've been alive for a good while, and have lived for over half a
century. People my age are prone to thinking of this current state of
the world as something like this: "Well, it certainly is a brutal world,
but I guess that's how things go. A country like Japan seems peaceful on
the outside. Maybe we can just say there's nothing to worry about, and
leave it at that." "Well, what can we do now? We've already come so
far."

But even so, after meeting these boys who tear through the streets of
rubble, refusing a world ornate in artifice, attempting to face off
against a harsh reality, living each and every day as themselves---I come
to think there's no way I could gloss it over or simply give up after
all.

But with that said, I wonder what I could do, what I ought to do, and I
wrestle with my thoughts and can do nothing but hesitate in a nervous
limbo. Maybe I'm afraid to take that first step from fear of getting
hurt.

Ugh, I'm sure Nezumi is laughing at me right now.

Adults are free to make excuses and give up; no matter what consequences
arise, they will have no one to blame but themselves. But young men and
women don't have it quite the same. They must keep living and survive.
They cannot accept despair as easily.

To see the world at their side; to start off from a place in which I've
rejected despair; to grasp this world with words that are not false
trinkets---is it something I would be able to do?

I strongly hope to challenge myself and the reality around me, with No.
6 as my weapon. The chances of my winning are slim, but I'd like to
believe\el at least, that I won't be losing constantly.

My gratitude from the bottom of my heart to those who have read thus
far.

\myspace

\emph{Summer 2007\\
	Atsuko Asano}

\clearpage
\subsection{Volume IV}
\paragraph{Tankobon}

It may be a sheepish, foolish, and embarrassing thing to write only
about your most personal thoughts in a space like the afterword.
Thinking back, I realize I've repeated this blunder over and over again,
and even I get sick of it sometimes. So I think I will make this my
last. Will you put up with my complaints one last time? I'm sorry.

This year, I lost two people whom I was very close to. One was Mr. O'oka
Hideaki, a critic and fellow member of our coterie magazine; the other
was Mr. Yamakage Yoshikatsu, of Kodansha's Children's Books Department.
Both supported me as a writer from their respective positions in their
own ways. Being the crude individual that I am, I only realized after I
lost these two how much their support had meant to me, and in my loss,
confusion, and loneliness, I sobbed like a wandering child at sunset.

Mr. Yamakage particularly was my irreplaceable partner in creating the
story of \emph{No.~6}. He was someone who had stayed with me since Volume 1. He
was also the one who gave this story its title, \emph{No.~6}. And more than
anything, he has taught me what it means to live on, and what it means
to die.

The following are words that I can't forget.

It was either the beginning of summer, or the end---a time when the
seasons were changing. Mr. Yamakage and I were talking about
this-and-that of my next work inside a taxi, when he said:

"Ms. Asano, you know, these days I've been sweating."

Mr. Yamakage said this suddenly, lowering his voice a little. The
utterance had a hint of a smile in it, like he often used to speak. \emph{So?}
I thought. \emph{Sweat? Isn't it a normal thing to sweat when it gets hot?} I
must have had a bemused expression on my face from not understanding the
meaning behind his utterance. But he continued.

"When it's hot, I sweat like I should. It makes me think, wow, I'm
alive."

I realized that it had only been a short time since Mr. Yamakage had
returned to the workplace after recovering from his serious illness; I
nodded then, thoroughly convinced. And now, I contemplate those words
and feel the weight of them all the more.

---Because that is what being alive means. It's sweating when you're hot;
it's crying when you're sad; it's laughing when you're happy. It's
walking straight down a road, and climbing the stairs. It's the days
that pass by, ordinary, mundane, that prove that we are still alive. Mr.
Yamakage taught me that. \emph{No.~6} is a story of the boys. It is also a
story of life and death. To a writer like yours truly, who had been
trying to write about life and death as the crux of the story, yet at
the same time in a light and comedic way, perhaps Mr. Yamakage had
stepped beyond his bounds as an editor to convey this message to me.

\emph{Ms. Asano, please, truly love that you are alive; cherish it, and
preciously, preciously write about it. Let's make \emph{No.~6} that kind of
story---where real human 'life' resides.}

He was a brilliant man. He was not afraid at all to live his life
through, and fall into the clutches of death. I wish he could have run
this course with me for a little while---no, for the whole time.

Mr. Yamakage, you went too soon. It's not fair that you just disappeared
like that, engraving yourself in my memories. When I meet you in the
afterworld, I'll be sure to bombard you with complaints. And you'll
probably flash that smile, nod quietly, and apologize in that sheepish
way.

Thank you to everyone who has waited for Volume 4. And I apologize (for
publishing it much, much later than I had originally promised).

And when I was ready to fall to my knees, blurting that maybe this story
was finished too because Mr. Yamakage was gone, I thank everyone who
supported me: Mr. Abe Kaoru, and Mr. Yamamuro Hideyuki, who supported me
in his place; Mr. Kageyama Toru and Mr. Kitamura Takashi, who finished
their jobs like true professionals, and sent me vigorous encouragement
that needed no words. I thank you very, very much.

\myspace

\emph{August 2005\\
	Atsuko Asano}

\paragraph{Bunko}

It's an embarrassing story, but when I write afterwords, these days all
I seem to end up with are complaints or excuses. I think it is
absolutely necessary that every story---\emph{No.~6} as no exception---should
refuse any complaints or excuses.

For this reason, this time around I've decided to write not any sort of
afterword, but just my thoughts as they come to me.

While I was writing \emph{No.~6} Volume 4---or, rather, throughout this whole
series---I've been thinking about what "hope" is.

Hope is believing in the future.

In this world right now, did I really hold a firm belief in the future
as I was writing? I'm still thinking about it (since this series is
still going, after all).

I think and I think, but no matter how much I do, I can't seem to grasp
the answer.

It's not that I've lost hope. In this day and age, I do naturally feel a
sense of imminent danger, to an extent (though it may not be directed
accurately at the right things). But I'm not despairing, nor have I
given up. But if someone were to ask me how much true hope I've got in
my hands---then, well, I've got no choice but to tilt my head in
perplexity. It's certainly an uneasy story\el 

Hunger, warfare, destruction, poverty, murder, despair\el 

Change is occurring both on the surface and within people, and these
changes twist and turn; and in our every day lives, like people riding
on a flimsy boat of bamboo leaves in a swift current, we don't know when
we'll be sucked into the whirlpool.

The small light of hope that winked inside me while I was still writing
Volume 1 has now become hard even to make out with my degree of vision.

Has my eyesight gotten worse?

Or has the light gotten weaker?

Hmm? This is starting to sound a lot like a complaint. Note to self:
mind that it doesn't.

Stories detest and avoid complaints and excuses like nothing else. At
the same time, they encourage your struggle to believe in the future.

Stories will not develop or be born from anyone who says, "Well, that's
just how it is" with a skewed and pessimistic outlook; nor does it come
from those who have thrown everything away, saying, "I don't care what
happens anymore". Only those who squint at that tiny ray of light, and
take that hesitant half-step forward---only from that half-step is a story
born.

Perhaps believing in that half-step you take is somehow connected more
largely to believing in the future.

And to you, who has read this story thus far---let's take that hesitant
half-step forward together, why don't we?

\myspace

\emph{Summer 2008\\
	Atsuko Asano}

\clearpage
\subsection{Volume V}
\paragraph{Bunko}

This \emph{No.~6} series has finally reached its fifth volume. I still remember
complaining in Volume 1 how I was ashamed of myself for turning my
afterwords into excuses, and saying \emph{'I don't want to write them
anymore!'}. But after thinking it over calmly again, I realized that it
wasn't the \emph{afterwords} I didn't like; it was me---making excuses,
justifying myself with this or that---that I disliked. So basically, I'd
been taking my frustrations out on the afterword itself. I must confess,
that's not getting to the root of the problem at all. I'm sick of it,
really.

These days I really think that people like me---who are skilled in the art
of self-preservation, are cowardly, but also ambitious---shouldn't be
writing a story like \emph{No.~6}. I may have written a bit about this
somewhere else, but \emph{No.~6} to me as a work was something a little out of
the ordinary. To me, the core of a work was always in humans. I wanted
to write about, and know more about, none other than people. The only
device I had at my hands that would let me understand people was
writing. I wanted to know these girls, these boys, these men and women.
I wanted to know what kind of people they were. That was the energy
behind why I wanted to start writing, and it was the reason I kept
writing.

But before I started writing the story of \emph{No.~6}, I wished to know the
world before I started getting to know the people. I hoped for a story
that would help me face the world I was living in now. It was my first
experience. That was why at first, I was not so much interested in the
true form of Shion, or Nezumi---what they thought, what they loved, what
they loathed as they lived their lives. The Holy City was the
protagonist of this story, and the boys were only side characters. But
it wasn't long before those arrogant thoughts were shattered to pieces.
But of course: it was impossible to render a world in which humans were
neglected a place. People are always connected to the world. People are
what comprise the world itself. The world is created by people, who make
it bountiful, who make it corrupt, who destroy it, and bring it back to
life.

Before I knew it, I was the one desperately following Shion and Nezumi,
enchanted by the world they created, the changes they underwent, and
their fates. And though it took long enough, it finally hit home for me
that the only way to render this world was to follow them, watch them,
grasp them, and pen them. It was a reckless challenge. I feel like a
praying mantis brandishing its tiny claws at an enormous oncoming cart.\footnote{A Japanese idiom; one who enters danger heedless of one's own weaknesses.}
I don't have that resolve. I don't have the guts to face the world, or
my own self head-on. That was also what I realized while writing this
story. And as soon as I realized it, it hurt to hear Nezumi's words and
feel Shion's gaze. So now we've come to this: whatever shall I do? I
wish I could just throw it away\el Oh dear me, now instead of excuses
I'm griping. Hmm, not good. But I'll hang in there for a little more. If
I don't pull myself up by my bootstraps now, I wouldn't know what I'd
written this far for; so on and so forth, blah blah.

Thank you for supporting me and putting up patiently with my reckless
challenges and weak-willed excuses: Mr. Harada Hiroshi from the Bunko
Publishing Department; Mr. Yamashiro Hideyuki from the Children's
Publishing Department. And my heartfelt thanks to you, reader, who has
taken the time to read this work.

\myspace

\emph{Summer 2009\\
	Atsuko Asano}

\clearpage
\subsection{Volume VII}
\paragraph{Bunko}

Hello, everyone. Asano here. How did you find \emph{No.~6} Volume 7? To make an honest confession, Volume 7 was a volume that was incredibly difficult and painful to continue writing. I struggled to write, struggled to think; nothing moved forward, and while I was writing I was rocked by hesitation and an emotion similar to panic.

I don't mean that I was simply in a block (although a considerable fraction of it was). It was over ten years ago when I first started writing \emph{No.~6}. The first volume was published in 2003, and it has already been nine years since then. When I began writing, my heart was not so much with Shion and Nezumi, but with No.~6 itself. With a fictional city-state at centre-stage, I wanted to write about a state which ruthlessly trampled its people, and with my pen capture every scene of their domination over its people. I had that desire---no, ambition.

I have already finished writing the last volume of the hardcover, and put a period to this series, at least in form. But if you were to ask me if my ambitions were realized\el 

What is a state? How would a country and its people interact? What is the difference between the rulers and the ruled? They were themes much too large for me to tackle with my level of strength. I feel like I am still standing, completely at a loss of what to do before a thing of such magnitude.

However, as I continued to progress writing through this series, despite its sweeping theme, my heart was swept away by these two boys, Shion and Nezumi. I became compelled to grasp them firmly with my own two hands. No matter what anyone said, to me, they were both very attractive characters whom I believed deserved to be known. Before I knew it, I feel like I have stayed fixated on this series with the singular mission to complete writing, not the city, but these boys as they lived on, dashed about, jumped, fought, became attached to others, felt love, and felt hatred.

In that sense, you can say that this Volume 7 is the most meaningful (for me, at least) in the whole series. By infiltrating the Correctional Facility, both Shion and Nezumi lay bare a side of their selves which have before been lurking in their depths. I struggled to write because I agonized and hesitated about how to write this very part.

In the Correctional Facility, Nezumi and Shion are cornered, their movements inhibited at gunpoint.

\emph{I see. So I am going to die with you.}

When Nezumi muttered this phrase in his heart, I thought of putting the two out of their misery. They would be more at peace if they were pierced by a bullet together, I thought. Of course, that would do nothing for the story. The real reason that I chose to write further, however, has nothing to do with what "ought to be" in a story. It was my own conviction as a writer. To others, it was perhaps too insignificant, but to me it was an important thing. I felt that if I didn't write the rest of this story, my fixation with No.~6 would have been meaningless.

I can only leave it up to you to read it as you will interpret it. Volume 7 has become that kind of volume.

\myspace

\emph{Summer 2012\\
	Atsuko Asano}